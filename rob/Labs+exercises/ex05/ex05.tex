\documentclass[11pt]{article}
\usepackage{amsmath}
\usepackage{listings}
\usepackage{parskip}	%to remove paragraph indentation
\usepackage{float}	%to fix the image position
\usepackage{svg} % for using SVGs directly

%Gummi|065|=)
\title{\textbf{Statistical Methods for Bioinformatics [I0U31a] \\Assignment 05 - Chapter 7}}
\author{Hamed Borhani}
\date{\today}
\usepackage{graphicx}
\begin{document}
\lstset{language=R, breaklines=true, basicstyle=\footnotesize} 

\maketitle

\newpage
\paragraph{7.9.10 }

\subparagraph{(a)}
\subparagraph{(a)}
\begin{lstlisting}
library(ISLR)
library(leaps)
attach(College)

#train/testr
set.seed(2)
train <- sample(1:nrow(College), nrow(College)/2)

fwd.college <- regsubsets(Outstate~., data = College[train,], 
                          nvmax = 17, method = "forward")
> plot(summary(fwd.college)$cp, type = 'b')
> plot(summary(fwd.college)$bic, type = 'b')
> plot(summary(fwd.college)$adjr2, type = 'b')
\end{lstlisting}

\begin{figure}[H]
\centering
\includesvg[width=1\columnwidth, svgpath =/home/hamed/KUL/SEM2/stat/Rob/Labs+exercises/ex05/]{cp}
\caption{CP plot}
\label{}
\end{figure}
\begin{figure}[H]
\centering
\includesvg[width=1\columnwidth, svgpath =/home/hamed/KUL/SEM2/stat/Rob/Labs+exercises/ex05/]{bic}
\caption{BIC plot}
\label{}
\end{figure}
\begin{figure}[H]
\centering
\includesvg[width=1\columnwidth, svgpath =/home/hamed/KUL/SEM2/stat/Rob/Labs+exercises/ex05/]{adjr2}
\caption{Adjusted $R^2$}
\label{}
\end{figure}
Minimum CP and also maximum adjusted $R^2$ correspond to the model with 11 variables. But minimum BIC corresponds to the model with 9 variables. So we fit a model with this 11 variables:
\begin{lstlisting}
> coef(fwd.college, id = 11)
  (Intercept)    PrivateYes          Apps        Accept        Enroll 
-2266.1474174  2591.6010582    -0.2576897     0.8452096    -1.0789110 
    Top10perc    Room.Board      Personal      Terminal   perc.alumni 
   17.3710739     0.7653948    -0.3793180    27.0520396    37.8893471 
       Expend     Grad.Rate 
    0.2642780    28.2822643

lm.college <- lm(Outstate~ Private + Apps + Accept + Enroll 
                 + Top10perc + Room.Board + Personal + Terminal 
                 + perc.alumni + Expend + Grad.Rate, 
                 data= College, subset = train)

yhat.lm <- predict(lm.college, newdata = College[-train,])

> min((yhat.lm - College[-train, "Outstate"])^2)
[1] 0.1008338
\end{lstlisting}

\subparagraph{(b)}
\subparagraph{(b)}
\begin{lstlisting}
library(gam)

gam.college <- gam(Outstate~ Private + s(Apps, df=2) + s(Accept, df=2) 
                   + s(Enroll, df=2) + s(Top10perc, df=2) 
                   + s(Room.Board, df=2) + s(Personal, df=2) 
                   + s(Terminal, df=2) + s(perc.alumni, df=2) 
                   + s(Expend, df=2) + s(Grad.Rate, df=2), data= College, subset = train)
                   
> plot(gam.college, se=TRUE)
\end{lstlisting}

\begin{figure}[H]
\centering
\includesvg[width=1\columnwidth, svgpath =/home/hamed/KUL/SEM2/stat/Rob/Labs+exercises/ex05/]{gam01}
\label{}
\end{figure}
\begin{figure}[H]
\centering
\includesvg[width=1\columnwidth, svgpath =/home/hamed/KUL/SEM2/stat/Rob/Labs+exercises/ex05/]{gam02}
\label{}
\end{figure}
\begin{figure}[H]
\centering
\includesvg[width=1\columnwidth, svgpath =/home/hamed/KUL/SEM2/stat/Rob/Labs+exercises/ex05/]{gam03}
\label{}
\end{figure}
\begin{figure}[H]
\centering
\includesvg[width=1\columnwidth, svgpath =/home/hamed/KUL/SEM2/stat/Rob/Labs+exercises/ex05/]{gam04}
\label{}
\end{figure}
\begin{figure}[H]
\centering
\includesvg[width=1\columnwidth, svgpath =/home/hamed/KUL/SEM2/stat/Rob/Labs+exercises/ex05/]{gam05}
\label{}
\end{figure}
\begin{figure}[H]
\centering
\includesvg[width=1\columnwidth, svgpath =/home/hamed/KUL/SEM2/stat/Rob/Labs+exercises/ex05/]{gam06}
\label{}
\end{figure}

\subparagraph{(c)}
\subparagraph{(c)}
\begin{lstlisting}

yhat.gam <- predict(gam.college, newdata = College[-train,])

> min((yhat.gam - College[-train, "Outstate"])^2)
[1] 1.739645e-05
\end{lstlisting}
Test error for the GAM fit is very small, which shows that it's a good fit.
\newpage
\paragraph{Vervij Data}
\subparagraph{}
\begin{lstlisting}
library(tree)

#dummy coding : DM->1 , NODM->0
phenotype = ifelse(data$meta == "DM", 1, 0)
vijver <- cbind(phenotype,data[,-1])

vijver$phenotype <- as.factor(vijver$phenotype)

#train/test
train <- sample(1:nrow(vijver), nrow(vijver)/2)
vijver.test <- vijver[-train, "phenotype"]

#building a tree
vijver.tree <- tree(phenotype~., data = vijver[train,])

> summary(vijver.tree)
Classification tree:
tree(formula = phenotype ~ ., data = vijver[train, ])
Variables actually used in tree construction:
[1] "NM_003981"      "M27749"         "NM_017443"      "NM_016109"     
[5] "Contig42615_RC"
Number of terminal nodes:  6 
Residual mean deviance:  0.1633 = 14.37 / 88 
Misclassification error rate: 0.04255 = 4 / 94 

tree.pred <- predict(vijver.tree, newdata = vijver[-train,], type = 'class')
> table(tree.pred, vijver.test)
         vijver.test
tree.pred  0  1
        0 40 23
        1 14 17
\end{lstlisting}

\begin{figure}[H]
\centering
\includesvg[width=1\columnwidth, svgpath =/home/hamed/KUL/SEM2/stat/Rob/Labs+exercises/ex05/]{vijver_tree}
\label{}
\end{figure}

\begin{lstlisting}

\end{lstlisting}

\end{document}
