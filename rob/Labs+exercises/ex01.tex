\documentclass[11pt]{report}
\usepackage{amsmath}
\usepackage{listings}

%Gummi|065|=)
\title{\textbf{Statistical Methods for Bioinformatics [I0U31a] \\Assignment 01 - Chapter 5}}
\author{Hamed Borhani}
\date{\today}
\begin{document}

\maketitle

\newpage
\paragraph{1. }
\begin{equation}
Var(X+Y)=Var(X)+Var(Y)+2Cov(X,Y)
\end{equation}

\begin{equation}
Var(\alpha X)=\alpha^{2}Var(X)
\end{equation}

\begin{equation}
Cov(aX,bY)=abCov(X,Y)
\end{equation}
\\ According to these, we can infer:
\\ \begin{align*} 
Var(\alpha X + (1-\alpha)Y) &= Var(\alpha X) + Var((1-\alpha)Y) + 2Cov(\alpha X, (1-\alpha)Y)
\\ &= \alpha^{2}Var(X) + (1-\alpha)^{2}Var(Y) + 2\alpha(1-\alpha)Cov(X,Y)
\\ &= \alpha^{2}\sigma^{2}_{X} + \sigma^{2}_{Y} + \alpha^{2}\sigma^{2}_{Y} - 2\alpha\sigma^{2}_{Y} + 2\alpha\sigma_{XY} - 2\alpha^{2}\sigma_{XY}
\end{align*}
\\To find the minimum of $\alpha$, we should take the first derivative with respect to $\alpha$ and set it equal to zero:
\begin{align*}
\frac{d}{d(\alpha)}&(\alpha^{2}\sigma^{2}_{X} + \sigma^{2}_{Y} + \alpha^{2}\sigma^{2}_{Y} - 2\alpha\sigma^{2}_{Y} + 2\alpha\sigma_{XY} - 2\alpha^{2}\sigma_{XY}) = 0
\\ &\Rightarrow 2\alpha\sigma^{2}_{X} + 2\alpha\sigma^{2}_{Y} - 2\sigma^{2}_{Y} + 2\sigma_{XY} - 4\alpha\sigma_{XY} = 0
\\ &\Rightarrow 2\alpha(\sigma^{2}_{X} + \sigma^{2}_{Y} - 2\sigma_{XY} = 2\sigma^{2}_{Y} - \sigma_{XY}
\\ &\Rightarrow \alpha = \frac{\sigma^{2}_{Y} - \sigma_{XY}}{\sigma^{2}_{X} + \sigma^{2}_{Y} - 2\sigma_{XY}}
\end{align*}

\paragraph{4. }

\paragraph{5. }
\paragraph{5. }
\begin{lstlisting}[language=R, breaklines=true, basicstyle=\ttfamily]
library(ISLR)
set.seed(22)

attach(Default)

#a.
fit.lor <- glm(default ~ income + balance, family = "binomial")

#b.
#i.
dim(Default)[1] #1000
training <- sample(1000, 500)
#.ii
fit.lor.val <- glm(default ~ income + balance, family = "binomial", subset = training)
#iii.
probabilities <- predict(fit.lor.val, newdata = Default[-training, ], type = "response")
predict <- rep("No", length(probabilities))
predict[probabilities > 0.5] <- "Yes"
#iv.
error <- mean(predict != Default[-training, ]$default)

#c.
training <- sample(1000, 500)
fit.lor.val <- glm(default ~ income + balance, family = "binomial", subset = training)
probabilities <- predict(fit.lor.val, newdata = Default[-training, ], type = "response")
predict <- rep("No", length(probabilities))
predict[probabilities > 0.5] <- "Yes"
error1 <- mean(predict != Default[-training, ]$default)

training <- sample(1000, 500)
fit.lor.val <- glm(default ~ income + balance, family = "binomial", subset = training)
probabilities <- predict(fit.lor.val, newdata = Default[-training, ], type = "response")
predict <- rep("No", length(probabilities))
predict[probabilities > 0.5] <- "Yes"
error2 <- mean(predict != Default[-training, ]$default)

training <- sample(1000, 500)
fit.lor.val <- glm(default ~ income + balance, family = "binomial", subset = training)
probabilities <- predict(fit.lor.val, newdata = Default[-training, ], type = "response")
predict <- rep("No", length(probabilities))
predict[probabilities > 0.5] <- "Yes"
error3 <- mean(predict != Default[-training, ]$default)
\end{lstlisting}

The test error for 3 repetitions are similar, yet varying based on sampling.

\begin{lstlisting}[language=R, breaklines=true, basicstyle=\ttfamily]
#d.
training <- sample(1000, 500)
fit.lor.val <- glm(default ~ income + balance + student, family = "binomial", subset = training)
probabilities <- predict(fit.lor.val, newdata = Default[-training, ], type = "response")
predict <- rep("No", length(probabilities))
predict[probabilities > 0.5] <- "Yes"
error <- mean(predict != Default[-training, ]$default)
\end{lstlisting}

The test error is similar to previous models, therefore adding the additional variable is not helpful.

\paragraph{6. }

\begin{lstlisting}[language=R, breaklines=true, basicstyle=\ttfamily]
#a.
fit.lor <- glm(default ~ income + balance, family = "binomial")
summary(fit.lor)
\end{lstlisting}

Estimates of std error : \\intercept: 4.348e-01, income: 4.985e-06, balance: 2.274e-04

\begin{lstlisting}[language=R, breaklines=true, basicstyle=\ttfamily]
#b.
boot.fn <- function(data, index){
        return (coef(glm(default~income + balance, data = data, subset = index, family = "binomila")))
}
\end{lstlisting}

\begin{lstlisting}[language=R, breaklines=true, basicstyle=\ttfamily]
#c.
library(boot)
boot(Default, boot.fn, 5000)
\end{lstlisting}


\paragraph{8. }
\end{document}
